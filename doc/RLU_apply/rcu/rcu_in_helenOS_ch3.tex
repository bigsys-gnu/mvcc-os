\documentclass{beamer}
\title{Review of RCU algorithms}
\subtitle{Read-Copy Update in HelenOS}
\usetheme{CambridgeUS}
\usecolortheme{beaver}
\usepackage{kotex}
\usepackage{graphics}
\usepackage{booktabs}
\usepackage{listings}
\usepackage{local_dir}
\lstset{inputpath=\rcucode/helenOS/,
  basicstyle=\scriptsize,
  numbers=left,
  xleftmargin=10pt,
  tabsize=2,
}
\graphicspath{ {\rcufigure/helenOS/}}
\author{Chang-Hui Kim}

\begin{document}

\begin{frame}
  \titlepage
\end{frame}

%% -------------------------------------------------------------------------

\section{General purpose user space RCU}

%% -------------------------------------------------------------------------

\begin{frame}[t]
  \frametitle{General purpose user space RCU}

  \begin{itemize}
  \item separate readers into two groups (prev, cur).
  \item each group has a counter for the number of readers.
  \item \texttt{Rcu\_synchronize} waits for both prev, cur group.
  \end{itemize}
  
\end{frame}

%% -------------------------------------------------------------------------

\begin{frame}[t]
  \frametitle{Pros}

  \begin{itemize}
  \item tolerates preemption in read-side critical section.
  \item need no kernel support.
  \end{itemize}
  
\end{frame}

%% -------------------------------------------------------------------------

\begin{frame}[t]
  \frametitle{Cons}

  \begin{itemize}
  \item readers have to issue a memory barrier both when entering and leaving
    read-side critical sections.
  \item grace period may be twice as long as it needs to be in the common case. 
  \end{itemize}
  
\end{frame}

%% -------------------------------------------------------------------------

\section{Signal based user space RCU}

%% -------------------------------------------------------------------------

\begin{frame}[t]
  \frametitle{Signal based user space RCU}

  \begin{itemize}
  \item similar to General URCU.
  \item in the read side critical section, Memory-barrier will
    be called only within signal handler.
  \end{itemize}
  
\end{frame}

%% -------------------------------------------------------------------------

\begin{frame}[t]
  \frametitle{Pros}

  \begin{itemize}
  \item light weight read-side critical sections.
  \end{itemize}
  
\end{frame}

%% -------------------------------------------------------------------------

\begin{frame}[t]
  \frametitle{Cons}

  \begin{itemize}
  \item detecting grace periods disrupts readers.
  \item grace period may be twice as long as it needs to be in the common case.
  \item requires an implementation of signals.
  \end{itemize}

\end{frame}

%% -------------------------------------------------------------------------

\section{Classic kernel RCU}

%% -------------------------------------------------------------------------

\begin{frame}[t]
  \frametitle{Classic kernel RCU}

  \begin{itemize}
  \item every thread in the read-side critical section disables preemption.
  \item the writer thread registers call-back function.
  \item timer interrupt handler
    \begin{itemize}
    \item batches call-back functions.
    \item checks the current grace period ended.
    \item manages grace periods.
    \end{itemize}
  \end{itemize}
  
\end{frame}

%% -------------------------------------------------------------------------

\begin{frame}[t]
  \frametitle{Pros}
  
  \begin{itemize}
  \item low overhead reader sections.
  \item low overhead grace period detection.
  \end{itemize}

\end{frame}

%% -------------------------------------------------------------------------

\begin{frame}[t]
  \frametitle{Cons}

  \begin{itemize}
  \item requires a regular sampling tick.
  \item read-side critical sections may not be preempted.
  \item grace period detection is held back if threads are alloted larger time quanta
    that they spend in the kernel.
  \end{itemize}
  
\end{frame}

%% -------------------------------------------------------------------------

\section{Huston's RCU}

%% -------------------------------------------------------------------------

\begin{frame}[t]
  \frametitle{Huston's RCU}

  \begin{itemize}
  \item batches call-backs in \texttt{Rcu-Call} until a predefined number
    of queued callbacks is reached.
  \item \texttt{Rcu-Read-Unlock} announces quiescent states if a grace period
    is in progress.
  \end{itemize}

\end{frame}

%% -------------------------------------------------------------------------

\begin{frame}[t]
  \frametitle{Pros}

  \begin{itemize}
  \item no regular sampling tick.
  \item low latency grace period end detection. (use semaphore)
  \end{itemize}
  
\end{frame}

%% -------------------------------------------------------------------------

\begin{frame}[t]
  \frametitle{Cons}

  \begin{itemize}
  \item \texttt{Rcu-Read-Unlock} uses the atomic exchange instruction.
  \item still requires a timer.
  \item Non-preemptible reader sections.
  \end{itemize}
  
\end{frame}

%% -------------------------------------------------------------------------

\section{Podzimek's RCU}

%% -------------------------------------------------------------------------

\begin{frame}[t]
  \frametitle{Podzimek's RCU}

  \begin{itemize}
  \item the first reader notices the start of a new grace period on each CPU.
  \item announces a quiescent state in \texttt{Rcu-Read-Lock} or \texttt{Rcu-Read-Unlock}.
  \end{itemize}
  
\end{frame}

%% -------------------------------------------------------------------------

\begin{frame}[t]
  \frametitle{Pros}

  \begin{itemize}
  \item Low overhead read-sice critical sections with at most one memory barrier
    issued by each CPU per grace period.
  \item No regular sampling timer ticks are required.
  \item Does not needlessly wake idle CPUs up.
  \item Works in interrupt handlers.
  \item No instrumentation of the scheduler or the interrupt dispatching mechanism
    necessary.
  \end{itemize}
  
\end{frame}

%% -------------------------------------------------------------------------

\begin{frame}[t]
  \frametitle{Cons}

  \begin{itemize}
  \item Intrusiveness of the detection depends on cache coherency.
  \item Read-side critical section may not be preempted.
  \end{itemize}
  
\end{frame}


\end{document}
